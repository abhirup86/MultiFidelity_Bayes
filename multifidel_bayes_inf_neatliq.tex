\RequirePackage{fix-cm}
\RequirePackage[hyphens]{url}
\RequirePackage[final]{graphicx} % need to show figures in draft mode
\documentclass[aps,pre,nofootinbib,superscriptaddress,linenumbers,10pt, draft,tightenlines]{revtex4-1}


% Change to a sans serif font.
\usepackage{sourcesanspro}
\renewcommand*\familydefault{\sfdefault} %% Only if the base font of the document is to be sans serif
\usepackage[T1]{fontenc}
%\usepackage[font=sf,justification=justified]{caption}
\usepackage[font=sf]{floatrow}

% Rework captions to use sans serif font.
\makeatletter
\renewcommand\@make@capt@title[2]{%
 \@ifx@empty\float@link{\@firstofone}{\expandafter\href\expandafter{\float@link}}%
  {\textbf{#1}}\sf\@caption@fignum@sep#2\quad
}%
\makeatother

%\linespread{0.956}

\usepackage{listings} % For code examples
\usepackage[usenames,dvipsnames,svgnames,table]{xcolor}
\usepackage{amsmath}
\usepackage{amssymb}
\usepackage{graphicx}
\usepackage{dcolumn}
\usepackage{boxedminipage}
\usepackage[colorlinks=true,citecolor=blue,linkcolor=blue]{hyperref}
\usepackage[]{microtype}
\usepackage[obeyFinal]{todonotes}
\usepackage{import}
\usepackage{setspace, siunitx, amsmath,amsfonts, adjustbox,booktabs, cleveref}
%\usepackage{caption}
\usepackage{subcaption}
\usepackage{enumitem}

\setlistdepth{20}
\renewlist{itemize}{itemize}{20}

% initially, use dots for all levels
\setlist[itemize]{label=$\cdot$}

% customize the first 3 levels
\setlist[itemize,1]{label=\textbullet}
\setlist[itemize,2]{label=--}
\setlist[itemize,3]{label=*}


\usepackage{titlesec}
\setcounter{secnumdepth}{5}


% Units
\DeclareSIUnit\Molar{\textsc{m}}


% Comments
\newcounter{comment}
\newcommand{\comment}[2][]{%
% initials of the author (optional) + note in the margin
\refstepcounter{comment}%
{%
\setstretch{0.7}% spacing
\todo[inline, color={cyan!45},size=\small]{%
\textbf{\footnotesize [\uppercase{#1}\thecomment]:}~#2}%
}}

% Start supplementary sections

\newcommand{\beginsupplement}{%
        \onecolumngrid
        \setcounter{table}{0}
        \renewcommand{\thetable}{S\arabic{table}}%
        \setcounter{figure}{0}
        \renewcommand{\thefigure}{S\arabic{figure}}%
     }

\graphicspath{{figures/}}
\floatsetup[table]{capposition=top}
\begin{document}

%\documentclass[a4paper,12pt]{article}
%\usepackage[superscript,biblabel]{cite}
%\begin{document}

%%%%%%%%%%%%%%%%%%%%%%%%%%%%%%%%%%%%%%%%%%%%%%%%%%%%%%%%%%%%%%%%%%%%%%%%%%%%%%%%
% DOCUMENT
%%%%%%%%%%%%%%%%%%%%%%%%%%%%%%%%%%%%%%%%%%%%%%%%%%%%%%%%%%%%%%%%%%%%%%%%%%%%%%%%

\title{[OUTLINE]: Parameterization of Non-Bonded Classical Mechanics Potentials for Neat Organic Liquids using a Multi-fidelity Bayesian Inference Approach}

\author{Bryce C. Manubay} 
\email{bryce.manubay@colorado.edu}
\affiliation{University of Colorado}

\author{Michael R. Shirts}
\thanks{Corresponding author}
\email{michael.shirts@colorado.edu}
\affiliation{University of Colorado}

% Date
\date{\today}

%%%%%%%%%%%%%%%%%%%%%%%%%%%%%%%%%%%%%%%%%%%%%%%%%%%%%%%%%%%%%%%%%%%%%%%%%%%%%%%%
% ABSTRACT
%%%%%%%%%%%%%%%%%%%%%%%%%%%%%%%%%%%%%%%%%%%%%%%%%%%%%%%%%%%%%%%%%%%%%%%%%%%%%%%%


\emph{Keywords: force field; molecular dynamics simulation; parameterization; inference; metamodels}

\maketitle


%%%%%%%%%%%%%%%%%%%%%%%%%%%%%%%%%%%%%%%%%%%%%%%%%%%%%%%%%%%%%%%%%%%%%%%%%%%%%%%%
% PRELIMINARIES
%%%%%%%%%%%%%%%%%%%%%%%%%%%%%%%%%%%%%%%%%%%%%%%%%%%%%%%%%%%%%%%%%%%%%%%%%%%%%%%%

\section{Preliminaries}
Definitions
\begin{itemize}
\item $V$: Volume
\item $U$: Total energy (including potential and kinetic, excluding external energy such as due to gravity, etc)
\item $S$: Entropy
\item $N$: Number of particles
\item $T$: Temperature
\item $P$: Pressure
\item $k_B$: Boltzmann constant
\item $\beta$: $(k_B T)^{-1}$
\item $M$: Molar mass
\item $\rho$: Density ($M/V$)
\item $H$: Enthalpy 
\item $G$: Gibbs Free Energy (free enthalpy)
\item $A$: Helmholtz Free Energy
\item $u$: reduced energy
\item $f$: reduced free energy
\end{itemize}
\pagebreak
%%%%%%%%%%%%%%%%%%%%%%%%%%%%%%%%%%%%%%%%%%%%%%%%%%%%%%%%%%%%%%%%%%%
%INTRODUCTION
%%%%%%%%%%%%%%%%%%%%%%%%%%%%%%%%%%%%%%%%%%%%%%%%%%%%%%%%%%%%%%%%%%%

\section{Introduction}
\begin{itemize}
	\item MD as a critical research tool
	%MD would be great, but FF issues
	%Huge amount of work to get protein relatively correct
	\begin{itemize}
		\item Force fields that are transferable and quantitatively accurate are necessary for molecular simulation to be useful. \cite{villin,villin2,drug_discov}
	\end{itemize}
    \item Transferability and inaccuracy issues
    \begin{itemize}
    	\item Transferability of MD force fields, and particularly sets of force field parameters, is a current limitation 
    	      in the molecular simulation field.\cite{transferability1,transferability2,
    	      transferability3,transferability4}  
        \item Inaccurate and poorly parameterized force fields have been shown to grossly misrepresent molecular systems.
              \cite{ffcomp1,ffcomp2,robustness} 
        \item It has been shown that depending on the choice of force field, the same experiments for the same or similar systems can produce quantitatively
              different results, making the choice of force field far more important than it should be. \cite{ffcomp1,ffcomp2,ewen_comparison_2016,petrov_are_2014,guvench_comparison_2008}\\
    \end{itemize}
	\item Parameterization efforts
	\begin{itemize}
		\item Early 
		\begin{itemize}
			\item Until very recently, force fields have been parameterized manually, guided by the intuition of expert computational
			      chemists.\cite{parm94,tip3p,burger,law,combined,rational,aipar,charmm1,charmm2,mm1,mm2,mmff}
			\item Despite attempts at improvement, many of the functional forms and parameters of popular force fields remain mostly 
			      unchanged due to the lack of clear, systematic methods for updating them.\cite{unchanged,monticelli}
			\item Force fields like AMBER \textit{parm94} showed intuitive departure by shrinking parameter space with clever atom typing 
			      defined by expert computational chemists.\cite{parm94}
		\end{itemize}
	    \item Second Gen
	    \begin{itemize}
		    \item The parameterization of GAFF used a semi-automated genetic algorithm approach 
		          to select parameters.\cite{amber}
		    \item The parameterization of the rigid Tip4p-Ew model utilized a unique gradient assisted method. \cite{tip4pew}
		    \item An incredible amount of work over a long period of time has still been necessary to get biomolecular force fields somewhat
		          correct.\cite{GROMOS53A5}
	    \end{itemize}
        \item Current efforts
        \begin{itemize}
        	\item A few notable attempts, such as GAAMP and ForceBalance, have been made in recent years towards the 
        	      development of more automated and systematic force field parameterization methods.\cite{GAAMP,FB1,FB2,FB3} 
        	\item Each made important contributions to automated force field parameterization through clever use of objective function 
        	      optimization, exploiting a variety of fitting data and allowing exploration of functional forms. 
        	%\item Even with these more sophisticated
        	%optimization schemes human intervention is required to assign weights to different kinds of data when they are included 
        	%in the same objective function. 
        	%\item Additionally, molecular systems aren't necessarily uniquely defined by a single parameter set.
            %\item Something about parameter typing with SMIRKS
        \end{itemize}
    \end{itemize}
    \item Bayesian parameterization
    \begin{itemize}
    	\item Previous uses
    	\begin{itemize}
    		\item Bayesian inference provides a robust statistical framework for force field parameterization. 
    		      It has been shown that Bayesian approaches can be applied to a wide variety of data driven sciences.
    		      \cite{bayes1,bayes2,bayes3,bayes5,bayes6,bayes7,bayes8,bayes_coarse} 
    		%\item Bayesian methods have been used for balancing data to help minimize influence of oversampled populations and generate more robust predictive   
    		%      models\cite{bayes2} to recalibrating initial force estimates in coarse grained MD models to target atomistic MD and experimental data
    		%      \cite{bayes_coarse,bayes1}. 
    		\item Bayesian inference methods have also been applied for uncertainty quantification in MD as well as limited parameterization
    		      problems on simple Lennard-Jones systems. \cite{bayes4,UQMDrizzi,LJexpBayes}
    		% See if i can cite that unpublished MD UQ chapter from NIST
    	\end{itemize}
        \item Surrogate models/metamodels
        \begin{itemize}
        	\item Metamodeling has been critical in accelerating sampling driven processes which involve expensive calculations \cite{mbar}
        	\item Need to include citations for the Bayes inference MD parameterization by the stats guys at ETH \cite{LJexpBayes}
    	    \item COFFE papers \cite{CoffeFull}    	    
        \end{itemize}    
        \item What our ideas for parameterization are/paper overall thesis
        \begin{itemize}
        	\item \textbf{Through systematically testing different multi-fidelity likelihood estimation workflows, we have found an optimal process which 
        	      maximizes computational efficiency while yielding a force field nearly identical to that which would be produced
        	      by a parameterization utilizing solely MD simulation for optimization.}
            \item How can we combine different techniques, to find reasonable force fields in medium dimensionality in a computationally efficient manner
            
        	%\item Full automation of finding likely parameter consistent
        	%      with a set of experimental data
        	%\item Exploration and quantitative comparison of model forms
        	%      and mixing rules 
        	%\item Systematic error quantification
        	%\item INSERT SCHEMATIC DIAGRAM
        \end{itemize}    
    \end{itemize}
\end{itemize}
%%%%%%%%%%%%%%%%%%%%%%%%%%%%%%%%%%%%%%%%%%%%%%%%%%%%%%%%%%%%%%%%%%%%%%%%
%METHODS
%%%%%%%%%%%%%%%%%%%%%%%%%%%%%%%%%%%%%%%%%%%%%%%%%%%%%%%%%%%%%%%%%%%%%%%%
\section{Methods}
\begin{itemize}
	\item Simulation protocol
	\item What parameters?
	\begin{itemize}
		\item Non-bonded for cyclohexane and ethanol 
		\item Since we're switching to neat organic parameterization, I'm going to add a few more molecules,
		      but try to keep the number of parameters capped at 10 (chain alkanes, cyclic alcohols, etc.)
		% Need to decide which molecules
		\item 10
		\item Specific SMIRKS:
		\begin{itemize}
			\item $[\#8X2H1+0:1]$, $[\#6X4:1]$, $[\#1:1]-[\#6X4]$, $[\#1:1]-[\#8]$, 
			      $[\#1:1]-[\#6X4]-[\#7,\#8,\#9,\#16,\#17,\#35]$
		\end{itemize}
	\end{itemize}
    \item Property calculation
    \begin{itemize}
    	\item Densities
	% BCM: Comment out some of this because we're really only going to be calculating properites one way
	% BCM: may want to use molar volumes instead of densities so don't have to propogate simulation error to new property.
	%      Easier to propogate error in experiment and transform experimental property
    	Starting with the equation used to calculate the density experimentally, 
    	\begin{equation} \rho = \frac{M}{V} \end{equation}
    	We replace the average with the esemble estimate (calculated either directly, or with reweighting) to obtain: 
    	\begin{equation} \rho = \frac{M}{\langle V \rangle} \end{equation}
    	\paragraph{Derivative Estimate}
    	From the differential definition of the Gibbs free energy $dG = VdP -SdT + \sum_i \mu_i dN_i$ that V can be calculated from the Gibbs free energy as:
    	\begin{equation} V = \left( \frac{\partial G}{\partial P} \right)_{T,N} \end{equation}
    	The density can therefore be estimated from the Gibbs free energy.
    	\begin{equation} \rho = \frac{M}{ \left( \frac{\partial G}{\partial P} \right)_{T,N}} \end{equation}
    	The derivative can be estimated using a central difference numerical method utilizing Gibbs 
    	free energies \\ reweighted to different pressures.
    	\begin{equation} \left( \frac{\partial G}{\partial P} \right)_{T,N} \approx \frac{G_{P + \Delta P} - G_{P-\Delta P}}{2\Delta p} \end{equation}
    	The density can then finally be estimated.
    	%MRS: I changed a lot of the lower case p's for pressure to upper case P's to use more standard symbols -- you should review and correct the rest.
    	\begin{equation} \rho \approx \frac{M}{\frac{G_{P + \Delta P} - G_{P-\Delta P}}{2\Delta P}} \end{equation}
    	This can be calculated from the reduced free energy $f$ if desired by simply substituting:
    	\begin{equation} \rho \approx \frac{\beta M}{\frac{f_{P + \Delta P} - f_{P-\Delta P}}{2\Delta P}} \end{equation}
    	
    	\subsubsection{Molar Enthalpy}
    	This section is on the relation of enthalpy to Gibbs free energy (should we need it). This is not an experimental quantity, but will be helpful in calculating related properties of interest. The enthalpy, $H$, can be found from the Gibbs free energy, $G$, by the Gibbs-Helmholtz relation: 
    	
    	\begin{equation}H=-T^2 \left(\frac{\partial \big(\frac{G}{T}\big)}{\partial T}\right)_{P,N}\end{equation}
    	
    	Transforming the derivative in the Gibbs-Helmholtz relation to be in terms of $\beta$ instead of $T$ yields:
    	
    	\begin{equation}H=-T^2  \frac{\beta^2}{\beta^2}\left(\frac{\partial \big(\frac{G}{T}\big)}{\partial T} \frac{\partial T}{\partial \beta} \frac{\partial \beta}{\partial T}\right)_{P,N}\end{equation}
    	
    	
    	Recall that $\beta = \frac{1}{k_B T}$, therefore $\frac{\partial \beta}{\partial T} = - \frac{1}{k_B T^2}$. Substituting these values into the enthalpy equation gives:
    	
    	%MRS: usually, you should just use \left( instead of \big(, since it will choose the correct size for you! 
    	\begin{multline}
    	H = \frac{1}{k_B^3 T^2 \beta^2} \left(\frac{\partial \big(\frac{G}{T}\big)}{\partial \beta}\right)_{P,N}  = \frac{1}{k_B} \left(\frac{\partial \big(\frac{G}{T}\big)}{\beta}\right)_{P,N} = \frac{\partial f}{\partial \beta}_{P,N} 
    	\end{multline}\\*
    	%MRS: I like this equation more, as it's hard to take finite differences of $G$ (as discussed above).  Then you can eliminate the last few lines.
    	
    	%MRS: Also, not clear it should go here because molar enthalpy isn't an experimental measurement.  OR explicitly mention it isn't an experimental quantity, BUT you need it for other quantites (like the enthalpy of mixing, and the heat capacity).
    	
    	%------------------------------------------------------------------------------------------------------------------------
    	
        \subsubsection{Speed of Sound}
	% Change back to heat capacity and look into calculating Cp using finite temperature differences with MBAR
	% Can calculate covariances using something callable in MBAR, but it's not documented. Can also do with bootstrap
	% and we know that's correct until we can validate the MBAR part. Can also use fluctuation calculations and bootstrap 
	% which will probably be the absolute easiest, but te MBAR estimates of uncertainty won't be good
        The definition of the speed of sound is\cite{sos}:
        \begin{equation}c^2 = \left(\frac{\partial P}{\partial \rho}\right)_{S} = -\frac{V^2}{M}\left(\frac{\partial P}{\partial V}\right)_{S}\end{equation}
        
        %MRS: I'd make the derivation more explicit, breakin down (dP/dV)_S -> in terms of the triple product rule, and then expressing it a a function of explicit deriatives, so it's clear what evaluations would be needed. 
        
        \begin{equation}c^2 = \frac{V^2}{\beta M}\left[\frac{\left(\frac{\gamma_V}{k_B}\right)^2}{\frac{C_V}{k_B}} + \frac{\beta}{V \kappa_T}\right]\end{equation}\\*
        
        Where:\\*
        \begin{equation}\gamma_V = \left(\frac{\partial P}{\partial T}\right)_{V}\end{equation}\\*
        
        $\gamma_V$ is known as the isochoric pressure coefficient. $\kappa_T$ is the same isothermal compressibility from equation 20\\*
        
        %BCM: Did this derivation myself, pretty sure it's right
        An alternate derivation, applying the triple product rule to $\left(\frac{\partial P}{\partial V}\right)_{S}$ yields the following.
        
        \begin{equation}\left(\frac{\partial P}{\partial V}\right)_{S} = \frac{\left(\frac{\partial S}{\partial V}\right)_{P}}{\left(\frac{\partial S}{\partial P}\right)_{V}}\end{equation}\\*
        
        \begin{equation}\left(\frac{\partial S}{\partial V}\right)_{P} = \left(\frac{\partial S}{\partial T}\right)_{P} \left(\frac{\partial T}{\partial V}\right)_{P} = \frac{C_P}{T} \left(\frac{\partial T}{\partial V}\right)_{P} = \frac{C_P}{T V \alpha}\end{equation}\\*
        
        Where $\alpha = \frac{1}{V} \left(\frac{\partial V}{\partial T}\right)_{P} = \left(\frac{\partial \ln V}{\partial T}\right)_{P}$ is the coefficient of thermal expansion. The second term in our triple product rule expansion, $\left(\frac{\partial S}{\partial P}\right)_{V}$, can be expressed as follows:
        
        \begin{equation}\left(\frac{\partial S}{\partial P}\right)_{V} = \left(\frac{\partial S}{\partial T}\right)_{V} \left(\frac{\partial T}{\partial P}\right)_{V} = \frac{C_V}{T} \left(\frac{\partial T}{\partial P}\right)_{V} = \frac{C_V}{T \gamma_V}\end{equation}\\* 
        
        Thus our derivation yields:
        \begin{equation}\left(\frac{\partial P}{\partial V}\right)_{S} = \frac{C_P \gamma_V}{C_V V \alpha}\end{equation}\\*
        
        Horn et al set out several ways for calculating $\alpha$\cite{horn}:
        
        \paragraph{Analytical derivative of density with respect to temperature}
        \begin{equation}\alpha = -\frac{d\ln\langle \rho \rangle}{dT}\end{equation}
        
        \paragraph{Numerical derivative of density over range of T of interest}
        The same finite differnces approach as shown for isothermal compressibility can be applied here, thus:
        \begin{multline}
        \alpha = -\frac{d\ln\langle \rho \rangle}{dT} = \\ -\frac{1}{2\rho(T,P)} \left(\ln \langle \rho(P,T+\Delta T)\rangle - \ln \langle V(P,T-\Delta T)\right\rangle)\end{multline}
        \paragraph{Using the enthalpy-volume fluctuation formula}
        \begin{equation}\alpha = \frac{\langle VH \rangle - \langle V \rangle \langle H \rangle}{k_B \langle T \rangle^2 \langle V \rangle}\end{equation}\\*
        
        Finite differences approximations and/or analytical derivation can also be used to calculate $\gamma_V$ or by note of the relation:
        \begin{equation}\gamma_V = - \frac{\alpha}{\kappa_T}\end{equation}\\*
    	
        %--------------------------------------------------------------------------------------------------------------------------
        
    \end{itemize}
    \item Methods for metamodeling
    \begin{itemize}
    	\item MBAR
    	\item Surrogate models
    	\begin{itemize}
    		\item Physically motivated models
    		\begin{itemize}
    			\item Density
    			\begin{equation}
    			\frac{D_{new}}{D_{old}} = \frac{{\sum_i \sum_{j \neq i} N_i N_j \left(\sigma_i \sigma_j \right)^{\frac{3}{2}}}_{new}}{{\sum_i \sum_{j \neq i} N_i N_j \left(\sigma_i \sigma_j \right)^{\frac{3}{2}}}_{old}}
    			\end{equation}
    			\item Enthalpy
    			\begin{equation}
    			\frac{H_{new}}{H_{old}} = \frac{{\sum_i \sum_{j \neq i} N_i N_j \left(\epsilon_i \epsilon_j \right)^{\frac{1}{2}}}_{new}}{{\sum_i \sum_{j \neq i} N_i N_j \left(\epsilon_i \epsilon_j \right)^{\frac{1}{2}}}_{old}}
    			% Rich pair-correlation function https://www.researchgate.net/profile/Richard_Messerly2
    			% SAFT (look into) http://pubs.acs.org/doi/abs/10.1021/ie0102201
    			% Reith papers again
    			% Cite how they differ
    			\end{equation}
    		\end{itemize}
            \item GP models
	    % Expand for our specific case. What precise cov matrices are we calculating?
            \begin{itemize}
            	\item Formalism for estimating some quantity $Z$ at
            	      unknown location $x_0$ ($Z\left(x_0\right)$) 
            	      from N pairs of observed values 
            	      $w_i\left(x_0\right)$ and $Z\left(x_i\right)$ where
            	      $i = 1,...,N$
                \item \begin{equation} \hat{Z}\left(x_0\right) =
                      \sum_{i=1}^N w_i\left(x_0\right) \times
                      Z\left(x_i\right) \end{equation}
                \item We find our weight matrix, \textbf{W}, by minimizing \textbf{W} subject to the following system of equations:
                \begin{itemize}
                	\item \begin{align}
                	&\underset{W}{\text{minimize}}& & W^T \cdot \operatorname{Var}_{x_i} \cdot W - \operatorname{Cov}_{x_ix_0}^T \cdot W - W^T \cdot \operatorname{Cov}_{x_ix_0} + \operatorname{Var}_{x_0} \\
                	&\text{subject to}
                	& &\mathbf{1}^T \cdot W = 1
                	\end{align}
                	\item where the literals \begin{equation}\left\{\operatorname{Var}_{x_i}, \operatorname{Var}_{x_0}, \operatorname{Cov}_{x_ix_0}\right\}\end{equation} stand for \begin{equation}
                	\left\{\operatorname{Var}\left(\begin{bmatrix}Z(x_1)&\cdots&Z(x_N)\end{bmatrix}^T\right), \operatorname{Var}(Z(x_0)), \operatorname{Cov} \left(\begin{bmatrix}Z(x_1)&\cdots&Z(x_N)\end{bmatrix}^T,Z(x_0)\right)\right\}\end{equation}
                \end{itemize}
                \item The weights summarize important procedures of the 
                      inference process:    	
                \begin{itemize}
                	\item They reflect the structural closeness of 
                	      samples to the estimation location, $x_0$
                	\item They have a desegregating effect, to avoid
                	      bias caused by sample clustering
                \end{itemize}         
            \end{itemize}
    	\end{itemize}
        \item \textbf{Hypothesis: With a multi-fidelity hierarchical observable calculation scheme, we can quickly approach the true forward model produced 
        	          by MD simulation}
          % Rework hypothesis: With hierarchical modeling we can quickly approach the true forward model produced by simulation
          
        \item Explanation of potential multi-fidelity posterior sampling algorithms:
        \begin{itemize}
        	\item 3 Levels of property calculation
        	\begin{itemize}
        		\item High fidelity: Full MD simulation at a single point in parameter space
        		\item Medium fidelity: Use MBAR to estimate properties over a conservative range of parameter space in order to create a hypervolume of data
        		      over which we can construct a model
        		\item Low fidelity: Use data from medium fidelity calculations in order to fit a regression model over a hypervolume of parameter space
        		\begin{itemize}
        			\item For right now, most plausible technique is GP regression, but could brainstorm some others
        		% Could just do this
        		% What do marginals around maxima look like as a function of iteration
        		% Constraints on proposal for using GP in order to keep from moving into regions of high uncertainty
        		\end{itemize}
        	\end{itemize}
            \item Using MBAR as a look up table (2 levels of property estimation)
            \begin{itemize}
            	\item High fidelity: Full MD simulation at a single point in parameter space
            	\item Medium fidelity: Use MBAR to estimate properties over a conservative range of parameter space in order to create a hypervolume of data
            	\begin{itemize}
            		\item Rather than attempting to fit a model we can use MBAR discrete MBAR calculated observables in order to evaluate our likelihood
            		\item Using a discrete set of calculations, we can iterate over those values in order to find the highest point of probability and then 
            		      perform a new simulation and repeat the process
            		\item As we narrow in on the region of highest probability density, we can refine the grid over which we're searching in order to more
            		      accurately represent the final posterior
            % Could be useful in low dimensionality and to start with
            	\end{itemize}
            \end{itemize}
        \end{itemize}
        %\item Construction of "gold standard" posterior distribution of force field parameters
        %\begin{itemize}
        %	\item Will need to brainstorm best way to go about doing this
        %	\item In theory, we could construct a posterior with purely simulation, but it would take so much resource time
        %	\item Might be a better idea to compare differences in the cheap posteriors we produce 
        	% Might need even to do this yet
        % Ideally we get to same minima every time
        % Multiple minima: some of the methods Rich developed could be useful for this 
        % Bias vs. uncertainty trade-off
        
        %\end{itemize}              
    \end{itemize}
\end{itemize}
%%%%%%%%%%%%%%%%%%%%%%%%%%%%%%%%%%%%%%%%%%%%%%%%%%%%%%%%%%%%%%%%%%%%%%%%%
%PROPOSED EXPERIMENTS AND ANALYSES
%%%%%%%%%%%%%%%%%%%%%%%%%%%%%%%%%%%%%%%%%%%%%%%%%%%%%%%%%%%%%%%%%%%%%%%%%
\section{Experiments + Results and Analyses}
\begin{itemize}
	\item \textbf{Hypothesis: Using a multi-fidelity likelihood calculation scheme described in the previous section will provide not only a substantial
	      speed up over a traditional inference approach with purely simulation used in the likelihood estimate, but 
	      will also result in a final force field with accuracy rivaling that of the expensive approach.}
	\item Experiments for testing sampling workflow
	%\begin{itemize}
	%    \item Do we include MBAR?
	%    \begin{itemize}
	%    	\item Use MBAR like a look up table. Don't fit points produced with MBAR. Just MC sample on a grid of points produced with MBAR to see 
	%    	      where the higher probability region is. If on edge, move to edge and simulate again, repeat.
	%    \end{itemize}
	%    \begin{itemize}
	%    	\item Note: Physical surrogates won't make use of MBAR since 
	%    	      there won't be fitting data
	%    \end{itemize}        
	%\end{itemize}
    \item \textbf{Hypothesis: Different sampling algorithms will affect the speed of convergence to our final force field as well final distribution of parameters
          sampled.}
    \item Ideas for comparing force fields resultant from different sampling methods
    % Robustness of parameter choices given a final posterior distribution
    \begin{itemize}
    	\item KL divergence
    	\item Speed of convergence
    	\item Stability of solution (from different starting points do we get same answer?) 
    	\item Simulation of properties, using final parameters, that 
    	      were not in training set
    	\begin{itemize}
    		\item 3-fold verification
    		\begin{itemize}
    			\item Different properties not in training set
    			\item Extrapolation to thermodynamic state outside of training set (T, P)
    			\item Other molecules outside of training set that have the same SMIRKS types
    		\end{itemize}    		
    	\end{itemize} 
    \end{itemize}
\end{itemize}
%%%%%%%%%%%%%%%%%%%%%%%%%%%%%%%%%%%%%%%%%%%%%%%%%%%%%%%%%%%%%%%%%%%%%%%%%%
%%%%%%%%%%%%%%%%%%%%%%%%%%%%%%%%%%%%%%%%%%%%%%%%%%%%%%%%%%%%%%%%%%%%%%%%%%%%%%%%
% BIBLIOGRAPHY
%%%%%%%%%%%%%%%%%%%%%%%%%%%%%%%%%%%%%%%%%%%%%%%%%%%%%%%%%%%%%%%%%%%%%%%%%%%%%%%%

\bibliographystyle{achemso} 
\bibliography{bayes1_manuscript}

\end{document}
